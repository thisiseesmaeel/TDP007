\documentclass{mall}

\newcommand{\version}{Version 1.0}
\author{Ahmed Sikh, \url{ahmsi881@student.liu.se}\\
  Sayed Ismail Safwat, \url{saysa289@student.liu.se}\\
  }
\title{Gruppkontrakt}
\date{\today}
\rhead{Sayed Ismail Safwat\\
Ahmed Sikh}



\begin{document}
\projectpage


\section{Förutsättningar}
\label{prereq}
\textbf{Ismail:}
Har svårt att komma igång med uppgifter och även hålla fokus. Behöver tydliga instruktioner över arbetsuppgifterna. Arbetar när man sitter tillsammans online.

\textbf{Ahmed:}
Har svårt att komma igång med uppgifter och även hålla fokus. Behöver tydliga
instruktioner över arbetsuppgifterna. Behöver tillgång till stor skärm men om
jag sitter hemma då är det lugnt.

\section{Hur vi arbetar tillsammans}

Vi kommer att arbeta på vardagar mellan klockan 8 - 17, det kan även bli längre om så krävs. Våran arbetsplats kommer att vara online. Vi kommer försöka med 45 minuters intervaller och sedan ta 15 minuter paus, vi en timme lunchrast vid klockan 12. Förutom det har vi även kommit överens om att vi tar med fika på fredagar. För att vi ska gör vårt arbete så effektivt som möjligt har vi bestämt att göra svårare uppgifter tillsammans, annars ska vi dela upp uppgifterna. Varje morgon funderar vi över vad som behövs göras och under det mötet delar vi upp arbetsuppgifterna. För att vara säker på att en uppgift är klar så går vi noga igenom den för att se så att allt fungerar som det ska. Utanför vår arbetstid så kommer vi hålla kontakt med hjälp av discord, där kan vi ta upp funderingar som vi har.




\section{Om jag tycker att något inte fungerar}
Om någon skulle komma sent så får den gruppmedlemmen jobba ikapp den tiden, detta kan dock bero på hur mycket försenad individen i fråga är. Skulle någon komma väldigt mycket försent så är det självklart att det arbetet måste göras färdigt. Personen som är i tid får börja arbetet själv och sedan får man diskutera dagens upplägg när gruppen är samlad.  Ifall det finns någon speciell anledning till att man vet att man kommer komma sent är det viktigt att kontakta den andra gruppmedlemmen.

Om någon inte slutför sinna uppgifter så är det viktigt att en diskussion sker kring det. Ifall det skulle vara så att det känns som uppgifterna är ojämnt fördelade är det även här viktigt att vi diskuterar det, då kan vi omfördela arbetet för att det ska kännas jämnare. Om andra problem med en gruppmedlem uppstår så är det viktigt med ärlighet, men det ska ändå göras på ett snällt sätt. Om något görs bra i arbetet berömmer vi varandra och ger konstruktivkritik om något inte är gjort lika bra.

\section{Utvärdering}

Vid utsatt tid: utvärdera hur gruppkontraktet har följts, fundera på ifall något i kontraktet behöver ändras, eller om något nytt behöver läggas till.

\begin{itemize}
\item \textbf{När ska vi påminna oss om gruppkontraktet och utvärdera hur det fungerat?}

  Vid behov, går vi igenom gruppkontrakt tillsammans och gör eventuella ändringar om vi upptäker att någon del inte fungerar.

\end{itemize}

\end{document}